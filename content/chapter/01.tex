\chapter{Einführung}

Die Komm.ONE entwickelte die letzten Jahre vor allem monolithische Anwendungen mit Java. Durch die Transformation zu einer moderneren Plattform die Containerisierung unterstützt bzw. darauf ausgelegt ist, sollen Anwendungen in einer Microservicearchitektur entwickelt werden. Als Pilotprojekt wird hierbei ein Modul der Anwendung WIBAS als Microservice implementiert und auf der neuen Plattform bereitgestellt.

Während die letzten Jahre die Anwendungen meist mithilfe von Java Webanwendungen (JSF/JSP) erstellt wurden, erhalten die modernisierten Anwendungen ein separates Frontend mithilfe eines JavaScript-Frameworks. Da viele Anwendungen neu geschrieben werden müssen, lohnt es sich Gedanken darüberzumachen wie die Frontend-Entwicklung nachhaltig und wiederverwendbar gestaltet werden kann.

In dieser Arbeit wird das Vorgehen und die Methodik beschrieben wie dies umgesetzt werden kann.

\chapter{Verwendete Technologien}

In der Neuentwicklung der Heimarbeit-Komponente wird im Frontend das Framework Angular (v14) verwendet. Angular ist ein komponentenbasiertes Frontend Typescript-Framework von Google welches Open-Source entwickelt wird \cite{angular.17.01.2023}. Es wird sehr stark davon ausgegangen, dass in Zukunft vermehrt Angular eingesetzt wird. Die einzigen momentan verfügbaren Open-Source alternativen wäre React oder Vue.

In älteren Anwendungen die meist mithilfe von JSF (Jakarta Server Faces) geschrieben wurden, ist das CSS-Framework \textit{Prime} zum Einsatz gekommen (Primefaces). Dieses existiert für verschiedenste Technologien, unter anderem für Angular, React und Vue.

Prime soll auch für kommende Anwendungen weiterverwendet werden, es soll jedoch darauf geachtet werden, dass das Design einheitlich ist. Hierfür wurde vor dem Projektstart ein neuer Styleguide für die Komm.ONE entwickelt, der das Corporate Design beachtet und moderner wirkt.

Für die weitere Entwicklung kann davon ausgegangen werden, dass Prime weiterhin für ein Frontend-Framework eingesetzt wird und die Wiederverwendbarkeit soll auf der Ebene des CSS-Frameworks aufbauen. 

Ein wichtiges Tool hierfür ist das Themeing, welches Prime anbietet. Dabei ist die Grundstruktur und das Design des CSS-Frameworks voneinander getrennt und durch die Auswahl eines anderen Themes wird alles Grundlegende behalten, aber Abstände, Schriften und Farben werden angepasst. 

Ein einfacher Schritt der Wiederverwendbarkeit, wäre es ein eigenes Komm.ONE-Theme \textbf{einmal} zu schreiben und dies in neuen und ggf. auch alten Anwendungen einzubinden. Dabei ist das Endergebnis identisch, egal ob nun Angular, React, Vue oder JSF verwendet wird.

\chapter{Entwicklung eines Prime-Themes}

Ein Theme in Prime ist letztendlich eine große CSS-Datei mit Regeln für alle Komponenten. Ein neues Theme muss dabei nicht von $0$ geschrieben werden, sondern es kann ein schon vorhandenes Theme verwendet werden. Diese finden sich, wenn primeng installiert wurde, unter: \lstinline|node_modules/primeng/resources/themes/**/theme.css|. Dabei stehen 33 Themes kostenlos zur Verfügung \cite{Primeng.20.01.2023}. 

Um den Komm.ONE-Styleguide auf das Theme zu übertragen wurde zunächst im Projekt eine Demo-Seite erstellt. Auf dieser werden alle UI-Komponenten des Primeng-Frameworks gruppiert angezeigt. Um herauszufinden welche Styleklassen angepasst werden müssen, können mit den Entwicklertools im Browser die Abhängigkeiten nachgeschaut werden. Anschließend wurden die Klassen entsprechend der Stylevorgaben angepasst und im Ergebnis iterativ überprüft ob dieser mit den Vorgaben übereinstimmt und konsistent ist. 

Das am Ende vorliegende Theme kann nun in jeder Prime Anwendung verwendet werden um den Basis-Stil von Komm.ONE zu übernehmen. Dies funktioniert jedoch, nur wenn:
\begin{itemize}
    \setlength\itemsep{-1em}
    \item Das Prime-Framework richtig angewandt wird.
    \item Einzelne Komponenten nicht an anderer Stelle überschrieben werden
    \item Anwendungsspezifische Styles angepasst werden. 
\end{itemize}

Das Theme bietet nur einen stilistischen Rahmen, aber ein Grundlayout (Header, Sidebar, Footer, etc.) ist in Prime nicht vorgegeben und muss selbstständig aufgebaut werden. Da diese jedoch statisch sind, kann das Layout und der Style aus der Styleguide-Dokumentation (\href{https://confluence.komm-one.net/display/GWR/Styleguide+1.0.1-alpha}{https://confluence.komm-one.net/display/GWR/Styleguide+1.0.1-alpha}) entnommen werden.
